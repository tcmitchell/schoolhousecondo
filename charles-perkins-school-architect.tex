\documentclass{article}
\usepackage{geometry}
\usepackage{url}
\def\Dash{\thinspace---\thinspace{}}
\geometry{lmargin=1.25in, rmargin=1.25in, tmargin=1.25in, bmargin=1.25in}
\title{The 1891 Architect of Boston's Charles C. Perkins School\\
\normalsize (145 St.\ Botolph Street, corner of Cumberland Street)}
\author{David Walden}
\date{\today, revision}
\begin{document}
\maketitle

\begin{quote}
\noindent{}Harrison Henry Atwood (Boston City Architect, 1889--1891) designed the Charles
C. Perkins school building.  It was mostly constructed and was completed (summer 1892)
while Edmund~M. Wheelwright was City Architect (1891--1895).  During its design and
construction period and during the first year of its use, the school was known as the
primary school in the Prince school district. In 1894 it was named after Charles C.
Perkins, a Boston School Committee member from 1871--1884 and ``art critic, author,
organizer of cultural activities, and an influential friend of design and of music in
Boston.''
\end{quote}

\bigskip
\noindent The 1992 edition of the \textit{A.I.A. Guide to Boston Architecture} \footnote{2nd
edition, Susan and Michael Southworth, Globe Pequot Press, Old Saybrook, CT, 1992.} lists
Edmund Wheelwright as the architect of the Charles C. Perkins School.  The first edition
says the same thing, as presumably does the third edition.  Michael Southworth, co-author
of this architectural guide said\footnote{Email of May 25, 2013.} that he believes their
information about the 1891 architect came from the architect of the 1981 building
renovation, Graham Gund.

Wheelwright was a noted architect and was City Architect for Boston from 1891 to 1895.\footnote{(Douglass Shand Tucci \textit{Built in
Boston--City \& Suburb}, New York Graphic Society, Boston, 1978, pp.\ 195--197.}
Wheelwright designed many well known Boston area edifices, among them the buildings for
the Fire Department Headquarters (now Pine Street Inn), Massachusetts Historical Society,
Horticultural Hall, New England Conservatory of Music and Jordan Hall, and the Longfellow
Bridge and Lampoon Castle (home of the Harvard \textit{Lampoon}).

It would be nice to think of our schoolhouse building as being one of Wheelwright's
designs, especially since he was especially noted for his school buildings and wrote a
book about their design\footnote{\textit{School Architecture: A General Treatise for the Use of
Architects and Others}, Rogers \& Manson, Boston, 1901.} and his school buildings are
featured in the first volume of Francis Ward Chandler's book \textit{Municipal
Architecture in Boston, from Designs by Edmund M. Wheelwright, City Architect, 1891-1895}.
\footnote{Bates \& Guild company, Boston, 1898.}  However, the absence of any mention of the
Charles C. Perkins school building in Wheelwright's own book or in Chandler's book cast
doubt on the attribution of this school building to Wheelwright.footnote{Both books are on the
shelves at the Boston Athenaeum.}

Visits in 2013 to the Boston City Archives (where Boston School Committee documents from
the era in question are on file) and the Massachusetts Historical Society (where
Wheelwright's own signed copies of his four annual reports as City Architect reside)
revealed the following detailed history regarding the 1891 architect of the Charles C.
Perkins primary school at the corner of St.\ Botolph and Cumberland Streets in Boston.

\begin{enumerate}

\item In the years around 1900, there were a number of school ``divisions''
in Boston, each containing  ``districts.''  The Perkins schoolhouse was in the Prince
District, one of four districts in the Fourth Division (Central City).

\item Document 56 in the bound set of ``Documents of the City of Boston for the Year
1890'' is titled ``City of Boston Report of the City Architect in Relation to Plans for
School Buildings,'' is dated March 12, 1890, and is signed ``Harrison H. Atwood, City
Architect,'' and says, ``Plans for the following school buildings are in course of
preparation and upon which no contracts have been awarded: \dots\ Primary School,
Cumberland and St.\ Botolph streets \dots''

\item Harrison Atwood ceased to be City Architect on March 30, 1891.  Edmund Wheelwright
was the next City Architect.

\item City Architect reports from Edmund Wheelwright covering the years 1891--1894
never mention the Charles C. Perkins School, but they do mention the Prince Primary
School which is distinct from the Prince Grammar School at the corner of Newbury and
Exeter Streets [\textit{Annual Report of the Architect Department for the Year 1891},
Rockwell and Churchill, City Printers, Boston, 1892;  \textit{Annual Report of the
Architect Department for the Year 1892}, Rockwell and Churchill, City Printers, Boston,
1893; \textit{Annual Report of the Architect Department for the Year 1893}, Rockwell and
Churchill, City Printers, Boston, 1894;  \textit{Annual Report of the Architect
Department for the Year 1894}, Rockwell and Churchill, City Printers, Boston, 1895]. (The
Prince Grammar School was designed by a City Architect before either Atwood or
Wheelwright).

\item In Wheelwright's 1984 annual City Architect's report [Appendix C continued,
School Department], the Prince Primary School is listed as the fourth and last school
design of Harrison Henry Atwood.  A later chart in this volume [Appendix E, Financial
Statement Concerning Brick School-Houses Built From 1880--1895] lists the construction
dates from December 17, 1890, to April 15, 1892 (maybe construction didn't really get
going seriously until early 1891 accounting for the 1891 date given elsewhere for the
building).

\item Books by Herndon and Bacon [\textit{Men of Progress: One Thousand Biographical
Sketches and Portraits of Leaders in Business and Professional Life in the Commonwealth
of Massachusetts}, compiled under the supervision of Richard Herndon and edited by Edwin
M. Bacon, pp.\ 706--707; and \textit{Boston of Today:  A Glance at Its History and
Characteristics}, also by Herndon and Bacon, p.\ 130] describe Atwood's Prince Primary
School as being on St.\ Botolph Street at the corner of Cumberland Street. The May 4,
1989, nomination form by Leslie Larson and Kimberly Shilland for the Bowditch School,
designed by Atwood, for the National Register of Historic Places notes that the
Herndon-Bacon book(s) must mean the Charles C. Perkins school
[\url{http://www.jphs.org/victorian/bowditch-school.html}].


\item School Document No.\ 21--1892: Annual Report of the School Committee of the
City of Boston, 1892 says [page 4], ``These [schools] were completed during the summer
and occupied soon after vacation \dots{} a new Primary on St.\ Botolph street in the
Prince District \dots''


\item School Document No.\ 22--1893: Annual Report of the School Committee of the
City of Boston, 1893, as part of a chart of all of the schools in the city on page~187
lists for the Fourth District the  ``St.\ Botolph-Street School'' with Clare E. Fairbanks
as ``3d Asst.''


\item School Document No.\ 19--1894: Annual Report of the School Committee of the
City of Boston, 1894, in another chart of all of the schools in Boston lists on page~167
``Charles C. Perkins School'' with Clare E. Fairbanks as ``3d Asst.''


\item The book ``Documents of the School Committee of Boston for the Year 1913'' in
Appendix~E page~210, as part of a list of all the schools in the city and the provenance
of their names, says ``\thinspace `Charles C. Perkins' (name of building), `Prince'
(district), `1891' (year erected), and `Charles C. Perkins, member of the School
Committee from 1871 to 1884' (named for).''


\item From the above documents we can see that  the design of the schoolhouse at the
corner of St.\ Botolph and Cumberland Streets started before March of 1890, construction
began in December of 1890, the building was completed in the summer of 1892, it was
occupied that fall, and it apparently was named for Charles Perkins sometime in 1894.

\item The wikipedia entry for Charles C. Perkins reports him to have been notable
interpreter of the arts (music, art, etc.) in addition to his involvement with the School
Committee.  He died in 1886, so naming a school after him in in the early 1890s makes
sense.  The year 1894 would have been the 10th anniversary of his retirement from the
School Committee.

\item There was some scandal involving Harrison Atwood.  He was active in politics
since college. He was given the post of City Architect at the age of 26.  Atwood later
won elective office to both the state House of Representatives and the House of
Representatives in Washington, DC.).  He was dismissed as City Architect only two years
after being appointed to the post, at the time of a change of Boston Mayor when he was
accused of avoiding competitive bidding processes to direct contract awards, thus costing
Boston too much money to build its buildings.  Wheelwright was chosen to follow Atwood,
and one of Wheelwright's initial tasks was to evaluate how much extra Atwood's bidding
practices had cost the city. After Atwood's dismissal, there was a libel action
[\url{http://archive.org/stream/testimonyharris00atwogoog/testimonyharris00atwogoog_djvu.txt},
page 181 mentions the Prince primary school].

\item Like Edmund Wheelwright, Harrision Atwood did architectural work that
is venerated today.  The Bowditch School, Congress Street Fire Station, and Harvard
Avenue Fire Station, which he designed, are all in the National Register of Historic
Places. The Bowditch School has certain external similarities to the Charles C. Perkins
School.

\end{enumerate}
\bigskip
\noindent Acknowledgments.  The investigation relating to who designed our schoolhouse
was done at the Boston Athenaeum, Boston Public Library, Massachusetts Historical
Society, and Boston City Archives, as well as via Google and the Internet.  Liz Francis,
Thomas Lester, Peter Drummey, and others helped at MHS, particularly Liz; Kim Tenney
helped at the BPL; Marta Crilly helped at the Boston City archives; and the staff of the
Athenaeum were helpful, as always, with directions to Pilgrim Lower Hall where the
``extra large'' volumes of Chandler's book reside. Michael Southworth took the time to
answer my query about the source of the information in the \textit{A.I.A. Guide to Boston
Architecture} and passed my query on to architect Jason Kaldis, who has done extensive
research on Wheelwright.  Kaldis sent an email and excerpts from city documents that are
consistent with what I found in various city documents.  F.\,Washington Jarvis wrote an
article about Wheelwright for the \textit{Newsletter of the Roxbury Latin School} (April
2013, pp.~53--58) that didn't mention the Charles C. Perkins School and thus motivated my
search for whether or not Wheelwright was the designer of this school building.

\end{document}
